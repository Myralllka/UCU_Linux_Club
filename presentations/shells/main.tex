\documentclass[usenames,dvipsnames,10pt,aspectratio=169]{beamer} 
% Add option 'aspectratio=169' for 16:9 widescreen 
% Add option  'handout' to ignore animations
% If you have a smaller amount of text, feel free to also try '11pt'! / Jesper

\usepackage[utf8]{inputenc}
\usepackage{verbatim}
\usepackage{minted}
\usepackage{graphicx}
\usepackage{wrapfig}
\usepackage{geometry}
\usepackage{listings}
\usepackage{color, xcolor}
\usepackage[document]{ragged2e}
\usetheme{umu}

\usemintedstyle{monokai}

\usepackage{hyperref}
\hypersetup{
    colorlinks=true,
    linkcolor=ucugreyish,
    filecolor=ucured,
    urlcolor=ucublue,
}
\urlstyle{same}

\lstdefinestyle{codestyle}{
    commentstyle=\color{ucugreyish},
    keywordstyle=\color{codeblue},
    numberstyle=\tiny\color{codegrey},
    stringstyle=\color{codepurple},
    basicstyle=\ttfamily\footnotesize,
    breakatwhitespace=false,         
    breaklines=true,                 
    captionpos=b,                    
    keepspaces=true,                 
    numbers=left,                    
    numbersep=5pt,                  
    showspaces=false,                
    showstringspaces=false,
    showtabs=false,                  
    tabsize=2
}

% %%% Bibliography
% \usepackage[style=authoryear,backend=biber]{biblatex}
% \addbibresource{bibliography.bib}

% \DeclareNameAlias{author}{given-family}

% %%% Suppress biblatex annoying warning
% \usepackage{silence}
% \WarningFilter{biblatex}{Patching footnotes failed}

%%% Some useful commands
% pdf-friendly newline in links
\newcommand{\pdfnewline}{\texorpdfstring{\newline}{ }} 
% Fill the vertical space in a slide (to put text at the bottom)
\newcommand{\framefill}{\vskip0pt plus 1filll}

%%% Enter additional packages below (or above, I can't stop you)! / Jesper
\renewcommand{\proofname}{\sffamily{Proof}}

% presentation template slides usage
% \framecard[color (not working)]{textbuf}
% \framesplit{Header}{picture}{textbuf}
% \framepic{image}{text}

% \begin{frame}[fragile]
% \lstset{style=codestyle}
% \begin{lstlisting}[language=Python, caption=Python example]
% import numpy as np
    
% def incmatrix(genl1,genl2):
%     m = len(genl1)
%     n = len(genl2)
%     M = None #to become the incidence matrix
%     VT = np.zeros((n*m,1), int)  #dummy variable
    
%     #compute the bitwise xor matrix
%     M1 = bitxormatrix(genl1)
%     M2 = np.triu(bitxormatrix(genl2),1) 

% \end{lstlisting}
% \end{frame}

%%%%%%%%%%%%%%%%%%%%%%%%%%%%%%%%%%%%%%%%%%%%%%%%%%%%%%%%%%%%%%%%%%%%%%%%%%%%%%%%%%%%%
\title{Linux course}
\subtitle{Shell}
\date[\today]{\small\today}
\author[Morhunenko Mykola]{Morhunenko Mykola}
\institute{APPS@UCU}

\begin{document}

\begin{frame}
\titlepage
\end{frame}

\begin{frame}{\contentsname}
\setbeamercolor{background canvas}{bg=ucugrey}
\tableofcontents
\end{frame}

\section{What is Command Shell?}
\framecard{What is Command Shell?}

\begin{frame}{What is command Shell?}
\begin{itemize}
    \item Command Shell is a computer program, that provide the user with a (CLI) command line interface to control the computer using keyboard, without GUI (Graphical user interface), for communication with the Linux system
    \item If you are using Linux, you have definitely see the command prompt. Usually it looks like {\color{ucugreen}\$} or, probably, {\color{ucugreen} \text{[username@hostname path] }\$}
    \item From the very beginning it looks like GUI is faster, but it is totally false: CLI just have high entry threshold. But it allows to write scripts (files with shell commands) that can automate routine, which is impossible in GUI
    \item Much more programs provide only CLI. If you want to use servers, connect to other computers via ssh, to be a real programmer, you must know shell
    \item You can check your shell by the command {\color{ucugreen} \$ echo \$SHELL}
    \item Most likely you have {\color{ucugreen} bash}, the most popular and stable one
    \item If you are done, you can leave the shell with {\color{ucugreen} exit} command, or by pressing the {\color{ucugreen}Ctrl+d} in the terminal emulator window
\end{itemize}

\end{frame}

\section{Bash/Zsh/Fish}
\framecard{Bash/Zsh/Fish}

\begin{frame}{What is Bash}
\begin{itemize}
    \item Bash stands for "Bourne(born)-again-shell"
    \item It is default shell for most Linux distros
    \item POSIX standard have a full description of the shell. Bash implements all this features, plus something own, known as {\color{ucugreen} bashism}
\end{itemize}
\end{frame}

\begin{frame}{Different shells}
    \begin{itemize}
        \item Bash is a standard shell for the majority of  Linux distros, but it doesn't mean it is the best one
        \item 
    \end{itemize}
\end{frame}


\section{Paths}
\framecard{Paths}
\begin{frame}{Path}
    \begin{itemize}
        \item  {\color{ucugreen} Path} - one of the most important terms in this topic (and in understanding, how programs work)
        \item Paths can be absolute or relative
        \item Absolute
            \begin{itemize}
                \item {\color{ucugreen} /} - also known as  {\color{ucugreen} root path}, all paths that starts from it are {\color{ucugreen} absolute}. Other examples:
                \item {\color{ucugreen} /home/username}
                \item {\color{ucugreen} /usr/local/share/zsh/site-functions/}
            \end{itemize}
        \item Relative
            \begin{itemize}
                \item Relative paths don't starts from the root. They are relative with respect to the current path.
                \item {\color{ucugreen} .zshrc}
                \item {\color{ucugreen} Documents/UCULinux/presentations}
            \end{itemize}
    \end{itemize}
\end{frame}
\begin{frame}{Path}
    \begin{itemize}
        \item Names are case-sensitive, that means {\color{ucugreen} /home/UserName} and {\color{ucugreen} /home/username} are different names
        \item There are also special paths
        \begin{itemize}
            \item {\color{ucugreen} ./} stands for the current path
            \item {\color{ucugreen} ../} stands for the path one step back. For example, if the {\color{ucugreen} ./} is {\color{ucugreen} /home/username}, the {\color{ucugreen} ../} will be {\color{ucugreen} /home}
            \item {\color{ucugreen} ~/} stands for the directory of current user. For example, for {\color{ucugreen} root} user {\color{ucugreen} ~/} will be {\color{ucugreen} /root}, and for {\color{ucugreen} username} it will be {\color{ucugreen} /home/username}
        \end{itemize}
    \end{itemize}
\end{frame}

\section{Where am I? Who am I?}
\framecard{Where am I? Who am I?}



\section{Permissions}
\framecard{Permissions}

\end{document}